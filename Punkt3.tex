% Options for packages loaded elsewhere
\PassOptionsToPackage{unicode}{hyperref}
\PassOptionsToPackage{hyphens}{url}
%
\documentclass[
]{article}
\title{Punkt3}
\author{}
\date{\vspace{-2.5em}}

\usepackage{amsmath,amssymb}
\usepackage{lmodern}
\usepackage{iftex}
\ifPDFTeX
  \usepackage[T1]{fontenc}
  \usepackage[utf8]{inputenc}
  \usepackage{textcomp} % provide euro and other symbols
\else % if luatex or xetex
  \usepackage{unicode-math}
  \defaultfontfeatures{Scale=MatchLowercase}
  \defaultfontfeatures[\rmfamily]{Ligatures=TeX,Scale=1}
\fi
% Use upquote if available, for straight quotes in verbatim environments
\IfFileExists{upquote.sty}{\usepackage{upquote}}{}
\IfFileExists{microtype.sty}{% use microtype if available
  \usepackage[]{microtype}
  \UseMicrotypeSet[protrusion]{basicmath} % disable protrusion for tt fonts
}{}
\makeatletter
\@ifundefined{KOMAClassName}{% if non-KOMA class
  \IfFileExists{parskip.sty}{%
    \usepackage{parskip}
  }{% else
    \setlength{\parindent}{0pt}
    \setlength{\parskip}{6pt plus 2pt minus 1pt}}
}{% if KOMA class
  \KOMAoptions{parskip=half}}
\makeatother
\usepackage{xcolor}
\IfFileExists{xurl.sty}{\usepackage{xurl}}{} % add URL line breaks if available
\IfFileExists{bookmark.sty}{\usepackage{bookmark}}{\usepackage{hyperref}}
\hypersetup{
  pdftitle={Punkt3},
  hidelinks,
  pdfcreator={LaTeX via pandoc}}
\urlstyle{same} % disable monospaced font for URLs
\usepackage[margin=1in]{geometry}
\usepackage{color}
\usepackage{fancyvrb}
\newcommand{\VerbBar}{|}
\newcommand{\VERB}{\Verb[commandchars=\\\{\}]}
\DefineVerbatimEnvironment{Highlighting}{Verbatim}{commandchars=\\\{\}}
% Add ',fontsize=\small' for more characters per line
\usepackage{framed}
\definecolor{shadecolor}{RGB}{248,248,248}
\newenvironment{Shaded}{\begin{snugshade}}{\end{snugshade}}
\newcommand{\AlertTok}[1]{\textcolor[rgb]{0.94,0.16,0.16}{#1}}
\newcommand{\AnnotationTok}[1]{\textcolor[rgb]{0.56,0.35,0.01}{\textbf{\textit{#1}}}}
\newcommand{\AttributeTok}[1]{\textcolor[rgb]{0.77,0.63,0.00}{#1}}
\newcommand{\BaseNTok}[1]{\textcolor[rgb]{0.00,0.00,0.81}{#1}}
\newcommand{\BuiltInTok}[1]{#1}
\newcommand{\CharTok}[1]{\textcolor[rgb]{0.31,0.60,0.02}{#1}}
\newcommand{\CommentTok}[1]{\textcolor[rgb]{0.56,0.35,0.01}{\textit{#1}}}
\newcommand{\CommentVarTok}[1]{\textcolor[rgb]{0.56,0.35,0.01}{\textbf{\textit{#1}}}}
\newcommand{\ConstantTok}[1]{\textcolor[rgb]{0.00,0.00,0.00}{#1}}
\newcommand{\ControlFlowTok}[1]{\textcolor[rgb]{0.13,0.29,0.53}{\textbf{#1}}}
\newcommand{\DataTypeTok}[1]{\textcolor[rgb]{0.13,0.29,0.53}{#1}}
\newcommand{\DecValTok}[1]{\textcolor[rgb]{0.00,0.00,0.81}{#1}}
\newcommand{\DocumentationTok}[1]{\textcolor[rgb]{0.56,0.35,0.01}{\textbf{\textit{#1}}}}
\newcommand{\ErrorTok}[1]{\textcolor[rgb]{0.64,0.00,0.00}{\textbf{#1}}}
\newcommand{\ExtensionTok}[1]{#1}
\newcommand{\FloatTok}[1]{\textcolor[rgb]{0.00,0.00,0.81}{#1}}
\newcommand{\FunctionTok}[1]{\textcolor[rgb]{0.00,0.00,0.00}{#1}}
\newcommand{\ImportTok}[1]{#1}
\newcommand{\InformationTok}[1]{\textcolor[rgb]{0.56,0.35,0.01}{\textbf{\textit{#1}}}}
\newcommand{\KeywordTok}[1]{\textcolor[rgb]{0.13,0.29,0.53}{\textbf{#1}}}
\newcommand{\NormalTok}[1]{#1}
\newcommand{\OperatorTok}[1]{\textcolor[rgb]{0.81,0.36,0.00}{\textbf{#1}}}
\newcommand{\OtherTok}[1]{\textcolor[rgb]{0.56,0.35,0.01}{#1}}
\newcommand{\PreprocessorTok}[1]{\textcolor[rgb]{0.56,0.35,0.01}{\textit{#1}}}
\newcommand{\RegionMarkerTok}[1]{#1}
\newcommand{\SpecialCharTok}[1]{\textcolor[rgb]{0.00,0.00,0.00}{#1}}
\newcommand{\SpecialStringTok}[1]{\textcolor[rgb]{0.31,0.60,0.02}{#1}}
\newcommand{\StringTok}[1]{\textcolor[rgb]{0.31,0.60,0.02}{#1}}
\newcommand{\VariableTok}[1]{\textcolor[rgb]{0.00,0.00,0.00}{#1}}
\newcommand{\VerbatimStringTok}[1]{\textcolor[rgb]{0.31,0.60,0.02}{#1}}
\newcommand{\WarningTok}[1]{\textcolor[rgb]{0.56,0.35,0.01}{\textbf{\textit{#1}}}}
\usepackage{graphicx}
\makeatletter
\def\maxwidth{\ifdim\Gin@nat@width>\linewidth\linewidth\else\Gin@nat@width\fi}
\def\maxheight{\ifdim\Gin@nat@height>\textheight\textheight\else\Gin@nat@height\fi}
\makeatother
% Scale images if necessary, so that they will not overflow the page
% margins by default, and it is still possible to overwrite the defaults
% using explicit options in \includegraphics[width, height, ...]{}
\setkeys{Gin}{width=\maxwidth,height=\maxheight,keepaspectratio}
% Set default figure placement to htbp
\makeatletter
\def\fps@figure{htbp}
\makeatother
\setlength{\emergencystretch}{3em} % prevent overfull lines
\providecommand{\tightlist}{%
  \setlength{\itemsep}{0pt}\setlength{\parskip}{0pt}}
\setcounter{secnumdepth}{-\maxdimen} % remove section numbering
\ifLuaTeX
  \usepackage{selnolig}  % disable illegal ligatures
\fi

\begin{document}
\maketitle

\begin{Shaded}
\begin{Highlighting}[]
\FunctionTok{suppressPackageStartupMessages}\NormalTok{(\{}
\FunctionTok{library}\NormalTok{(PxWebApiData)}
\FunctionTok{library}\NormalTok{(dplyr)}
\FunctionTok{library}\NormalTok{(tidyverse)}
\FunctionTok{library}\NormalTok{(lubridate)}
\FunctionTok{library}\NormalTok{(REAT)}
\FunctionTok{library}\NormalTok{(readxl)}
\NormalTok{\})}
\NormalTok{knitr}\SpecialCharTok{::}\NormalTok{opts\_chunk}\SpecialCharTok{$}\FunctionTok{set}\NormalTok{(}\AttributeTok{echo =} \ConstantTok{FALSE}\NormalTok{, }\AttributeTok{include =} \ConstantTok{FALSE}\NormalTok{)}
\end{Highlighting}
\end{Shaded}

\hypertarget{i-denne-oppgaven-skal-vi-finne-sysselsetting-i-de-ulike-nuxe6ringene-etter-arbeidsstedskommune.}{%
\section{I denne oppgaven skal vi finne sysselsetting i de ulike
næringene, etter
arbeidsstedskommune.}\label{i-denne-oppgaven-skal-vi-finne-sysselsetting-i-de-ulike-nuxe6ringene-etter-arbeidsstedskommune.}}

Velger ut og endrer navn, Haugalandet

Muterer inn i Haugbo

Tar ut kommunene

\hypertarget{legger-her-til-en-total-for-haugalandet.}{%
\subsection{Legger her til en total for
Haugalandet.}\label{legger-her-til-en-total-for-haugalandet.}}

\hypertarget{sunnhordaland}{%
\section{Sunnhordaland}\label{sunnhordaland}}

\hypertarget{henter-inn-data-fra-oppgave-1-kommunene-i-sunnhordaland}{%
\subsection{Henter inn data fra oppgave 1, kommunene i
Sunnhordaland}\label{henter-inn-data-fra-oppgave-1-kommunene-i-sunnhordaland}}

\hypertarget{fra-kommune-nummer-til-kommune-navn}{%
\subsection{fra kommune nummer til kommune
navn}\label{fra-kommune-nummer-til-kommune-navn}}

\hypertarget{totalt-anasatte-i-hver-nuxe6ring-for-haugalandet}{%
\subsection{Totalt anasatte i hver næring for
Haugalandet}\label{totalt-anasatte-i-hver-nuxe6ring-for-haugalandet}}

Grafen ovenfor viser til antall ansatte i ulike næringer på Haugalandet.
Ut i fra grafen kan en se at Helse- og omsorgstjenester er næring med
høyest antall ansatte på Haugalandet.

\hypertarget{totalt-anasatte-i-hver-nuxe6ring-for-sunnhordland}{%
\subsection{Totalt anasatte i hver næring for
Sunnhordland}\label{totalt-anasatte-i-hver-nuxe6ring-for-sunnhordland}}

I figuren ovenfor fremstilles det antall ansatte i de ulike næringene i
Sunnhordland, her kan en se at helse- og soisaltjenester har flest
ansatte.

Under \emph{lokaliseringskvoienter} har vi sett på tre ulike sektorer
både på Haugalandet og for Sunnhordland. Vi valgte da ut industri,
helse- og sosialtjenester og varehandel, reperasjon av motorvogner. Det
ble valgt ut tre forskjellige år, da første året som er 2008, 2010 og
det siste året som var 2020. Aldersgruppen vi valgte var 15 til 74,
fordi vi ser at denne aldersgruppen inneholder høyest tall av ansatte.

\emph{Haugalandet}

Resultatet til sektoren \textbf{industri} Haugalandet ble på for årene
2008, 2010 og 2020 ble 1,527396811, 1,434975844 og 1,601509868. Alle
årene så sier resultatet at det er en lokalnæring på Haugalandet.
Haugesund, Karmøy og Tysvær er de tre største kommunene med flere
industri bedrifter, de har for eksempel Aibel, Hydro og Equinor. Disse
kan ses også på som basenæringer.

\textbf{Helse- og sosialtjenester} ble våre resultater for 2008, 2010 og
2020, slik 0,9962243996, 0,9816929097 og 1,026935838. For 2008 og 2010
kan en se ut i fra resultatene at det er basefaktor. Hvis en ser på
realiteten vil en si er en lokalnæring, som resultatet i 2020 sier.
Fordi det er sykehus i haugesund.

I \textbf{varehandel, reperasjon av motorvogner} ble resultatene for
2008, 2010 og 2020, 0,9372344155, 0,9562311403 og 0,9144987383. Her vil
en se at alle resultatene er nder 1 som tilsvarer at det er en
basisnæring.

\emph{Sunnhordaland}

For \textbf{industri} sektoren i Sunnhordland har vi brukt samme år, og
samme aldersgruppe, der våre resultater ble 2,039077786, 2,05242696 og
0. Her fikk vi 0 i 2020 fordi det var mengel på informasjon i år 2020. I
de to andre resultatene er tallet mye høyere en 1, vil det si at
industri sektoren på Sunnhordalan er en lokalnæring. Her vil det nok
være mye mer av lokalt arbeid.

\textbf{Helse- og sosialtjenester} ble våre resultater for 2008, 2010 og
2020, slik 1,893611307, 2,283353332 og 0 (mangel på informasjon for
2020). Her ser en at i 2008 var det et lavere resultat enn i 2010, men
begge årene gir et resultat på over 1 som sier at det er lokalnæring.

I \textbf{varehandel, reperasjon av motorvogner} ble resultatene for
2008, 2010 og 2020 var våre svar 1,104016994, 1,247585395 og 0. Slik som
de over er resultatet for 2020 en mangel på. Det er også lokalnæring
slik som de andre to sektorene.

I den romlige Gini-koeffisienten har vi valgt å se på samme sektorene
som vi brukte for å finne lokaliseringskovientene. Da årene 2008, 2010
og 2020. Aldersgruppen er 15-74 år og sektorene er industri, helse- og
sosialtjenester og varehandel, reperasjon av motorvogner. Ved måling av
Gini-koeffisienten måler vi hvor stor avstanden på ``rik'' og ``fattig''
i fylket, tallet skal ligge i mellom 0-1. 0 betyr at det er lik inntekt
som formue for alle i landet, og på 1 er det de tlik inntekt og formue
for en person.

\emph{Haugalandet}

Industri gir et resultat på 0,08818581974, som vil si at den er så å si
null og tilsvarer en nok så lik inntekt som formue innen sektoren
industi på Haugalandet.

Innen Helse- og sosialtjenster fikk vi svaret 0,05768709678. Der
resultatet blir det samme som i sektor industri. Der helse- og
sosialtjenesten har nok så lik formue som inntekt.

I varehandel, reperasjon av motorvogner var svaret vårt 0,03682287627,
som gir et enda lavere resultat enn de forje, men har samme betydning at
det er lik inntekt som formue innen den sektoren.

\emph{Sunnhordland} På grunn av mangel for informasjon i år 2020, er det
kun to år som er regnet med i denne delen.

Industri ga oss et resultat på 0,0568049578 der en ser at formue og
inntekt er nok så like for industri sektoren.

Resultatet til helse- og sosialtjenester er 0,0568049578 nok så likt
tall som på Haugalandet, som også lik inntekt som formue.

Innen varhandel, reperasjon av motorvogner i Sunnhordland var vårt
resultat 0,03182481794 og gir en lik inntekt som formue innen sektoren.

En kan se ut i fra alle resultatene så ga det oss en lik inntekt og
formue på alle sektorene vi valgte ut. i manglet et år på hver sektor i
Sunnhordland som kan ha gitt oss andre tall en hva den kunne blitt. Vi
antar at resultatene er gode og har gitt oss rett svar i forhold til å
måle Gini-koeffisienten.

\emph{RDI}

Under målet for mangfold av næringer, RDI, ser vi på samme sektorene som
ovenfor. Med RDI måler vi om den regionale næringen er lik den nasjonale
næringen, jo høyere RDI en er jo mer lik er den regionale næringen den
nasjonle næringen.

\emph{Haugalandet} Resultatet vi fikk i industri var 65,73415204, det
vil si at den regionale næringen(industri) er ikke så lik som den
nasjonal næringen(industri).

Når det kommer til næringen helse- og sosialtjenesten fikk et tall på
1498,289335 som tilsvarer en høy RDI, og vil si at sektoren helse- og
sosialtjenessten regionalt på Haugalandet er svært lik den nasjonale
næringen.

Innen varhandel, reperasjon av motorvogner fikk vi resultatet 356,211496
har et høyee RDI tall enn industri som tilsvarer at den er mær lik den
nasjonale næringen(arhandel, reperasjon av motorvogner) enne sektoren
industri er den nasjonale.

\emph{Sunnhordland} Industri næringen i Sunnhordland fikk vi resultatet
0,1231639904, som vil si at det er et veldig lav RDI. Det vil si at den
regionale industrien er ikke lik den najsoanle næringen.

Samme gjelder helse- og sosialtjenesten 0,1292582968 som også tilsvarer
ingen likhet med den nasjonale næringen.

Varhandel, reperasjon av motorvogner fikk et negativ resultat
−0,04424893169, som vil si at det er ingen likhet mellom den najsonale
og regionale næringen.

Sammenligne med resultatene for data etter arbeidssted, og kommenter
hvilken informasjon resultatene gir om romlig mobilitet i
arbeidsmarkedet.

Kommer når tinger er kommet på plasss\ldots{} Ble forsinkelser.

\end{document}
